\usepackage[authoryear,round]{natbib}
\usepackage{multirow}

\newcommand{\sheetnum}{%
	08
}
%\setcounter{section}{\sheetnum-3}
\newcommand{\tutorialtitle}{%
    Class Imbalance and RBF-Networks
}
\newcommand{\tutorialtitleshort}{%
	RBF
}
% for slides
\subtitle{\sheetnum \tutorialtitle}

%\maxdeadcycles=1000 % Workaround for ! Output loop---100 consecutive dead cycles because of too many figures

% The following use of algroithms does not work well with the notes:
%
%
%
%
% instead use the following for your algorithms:
%
%\begin{figure}[!t]
%\removelatexerror
%\begin{algorithm}[H]
    % your algo here
    %\label{alg:algolabel}
    %\caption{algocaption}
%\end{algorithm}
%\end{figure}
%\begin{algorithm}
% Below is the definition for the command \removelatexerror:
\makeatletter
\newcommand{\removelatexerror}{\let\@latex@error\@gobble}
\makeatother

\begin{document} %%%%%%%%%%%%%%%%%%%%%%%%%%%%%%%%%%%%%%%%%%%%%%%%%%%%%%%

\sheet{\sheetnum}{\tutorialtitleshort}

\ttopic{\tutorialtitle}

\columnratio{0.2,0.8}\textbf{}
\begin{paracol}{2}
%\setlength{\columnseprule}{0.1pt}
%\setlength{\columnsep}{5em}

\begin{rightcolumn}

% notes version will ignore it
\begin{frame}
\titlepage
\end{frame}

\begin{frame}
\tableofcontents
\end{frame}

\mode<all>
\section{Dealing with imbalanced data}

\subsection{Motivation}

\begin{frame}\frametitle{\subsecname}

\only<1>{
Proportion of class labels in the dataset is not uniform.
Not the same as ``easy vs. difficult'' classes. This is only about class frequency.
}

\slidesonly{\vspace{-5mm}}

\begin{figure}[ht]
     \centering
     \savebox{\imagebox}{
	 \includegraphics[width=0.28\textwidth]{img/hist_balanced}}%
     \begin{subfigure}[t]{0.28\textwidth}
         \centering
         \usebox{\imagebox}% Place largest image
         \caption{balanced $\approx 1:1$}
     \end{subfigure}
     \hspace{5mm}
     \begin{subfigure}[t]{0.28\textwidth}
         \centering
         \raisebox{\dimexpr.5\ht\imagebox-.5\height}{% Raise smaller image into place
         \includegraphics[width=0.99\textwidth]{img/hist_imbalanced}
         }
         \caption{highly unbalanced}
         \label{fig:linear}
     \end{subfigure}
\end{figure} 

\only<2>{     
A strong imbalance leads to the classifier learning a trivial solution that indeed minimizes the average cost over the training samples.
}
\only<2>{
\begin{equation}
E^T = \sum_{\alpha=1}^p e^{(\alpha)}
\end{equation}
}
\only<3>{
\begin{equation}
E^T = 
{\color{blue}\sum_{\beta\,\in D_{+}}^{|D_{+}|} e^{(\beta)}}
+
{\color{red}\sum_{\beta\,\in D_{-}}^{|D_{-}|} e^{(\beta)}}
\end{equation}

\mode<article>{
where set $D_{+}$ and $D_{-}$ are the subsets of data of only positive and negative samples, respectively:

\begin{equation}
{
\color{blue}
D_{+} := 
\Big\{ \left(\vec x^{(\alpha)}, \vec y^{(\alpha)}_{T} \right) \Big|\,y^{(\alpha)}_{T} > 0 \,\Big\}
\quad \text{(subset of positive samples)}
}
\end{equation}

and 

\begin{equation}
{
\color{red}
D_{-} := 
\Big\{ \left(\vec x^{(\alpha)}, \vec y^{(\alpha)}_{T} \right) \Big|\,y^{(\alpha)}_{T} \le 0 \,\Big\}
\quad \text{(subset of positive samples)}
}
\end{equation}

}

By always predicting ``0'' it will be correct 95\% of the time.
}
\end{frame}

\subsection{Confusion matrix}

\begin{frame}\frametitle{\subsecname}

The confusion matrix differentiates between the types of mistakes a classifier makes.

\begin{tabular}{ll|l|l|l}
\cline{3-4}
												  &          & \multicolumn{2}{c|}{Ground truth label}                        &  \\ \cline{3-4}
												  &          & \multicolumn{1}{c|}{Positive} & \multicolumn{1}{c|}{Negative}  &  \\ \cline{3-4}
												  &          & \multicolumn{1}{c|}{``Cat''} & \multicolumn{1}{c|}{``not Cat''}  &  \\ \cline{1-4}
\multicolumn{1}{|r|}{\multirow{2}{*}{Prediction}} & Positive & \textbf{T}rue \textbf{P}ositives & \textbf{F}alse \textbf{P}ositives &  \\ \cline{2-4}
\multicolumn{1}{|r|}{}                            & Negative &  \textbf{F}alse \textbf{N}egatives & \textbf{T}rue \textbf{N}egatives &  \\ \cline{1-4}
												  &          & \multicolumn{1}{c|}{$P$} & \multicolumn{1}{c|}{$N$}  &  \\ \cline{3-4}
\end{tabular}	


\end{frame}

\subsection{Performance metrics}

\begin{frame}\frametitle{\subsecname}

\mode<presentation>{
\begin{tabular}{ll|l|l|l}
\cline{3-4}
												  &          & \multicolumn{2}{c|}{Ground truth label}                        &  \\ \cline{3-4}
												  &          & \multicolumn{1}{c|}{Positive} & \multicolumn{1}{c|}{Negative}  &  \\ \cline{3-4}
												  &          & \multicolumn{1}{c|}{``Cat''} & \multicolumn{1}{c|}{``not Cat''}  &  \\ \cline{1-4}
\multicolumn{1}{|r|}{\multirow{2}{*}{Prediction}} & Positive & \textbf{T}rue \textbf{P}ositives & \textbf{F}alse \textbf{P}ositives &  \\ \cline{2-4}
\multicolumn{1}{|r|}{}                            & Negative &  \textbf{F}alse \textbf{N}egatives & \textbf{T}rue \textbf{N}egatives &  \\ \cline{1-4}
												  &          & \multicolumn{1}{c|}{$P$} & \multicolumn{1}{c|}{$N$}  &  \\ \cline{3-4}
\end{tabular}	

}

\begin{equation}
 \text{ \# of correct classifications } 
 \corresponds \text{ Accuracy }
 = \frac{T\kern-.4ex P+T\kern-.4exN}{P+N}
\end{equation}

\only<2>{
\slidesonly{
{\placeimage{12.5}{10.}{img/hist_imbalanced}{width=25mm}}
}

In our imbalanced example:
\begin{equation}
\frac{T\kern-.4exP+T\kern-.5exN}{P+N} = \frac{0+9500}{500+9500} = 0.95
\end{equation}

\notesonly{
Accuracy is a misleading metric when the classes are imbalance.
}
}

\only<3>{
Other metrics:

How many of the positive labels did we get:

\begin{equation}
\text{sensitivity (recall) }
 = \frac{T\kern-.4exP}{P}
\end{equation}

and how often were we correct, whenever we predicted positive?

\begin{equation}
\text{precision }
 = \frac{T\kern-.4exP}{T\kern-.4exP+F\kern-.5exP}
\end{equation}

Recall and precision can be combined into:

\begin{equation}
\text{F1 score }
 = 2 \cdot \frac{\mathrm{recall} \times \mathrm{precision}}{\mathrm{recall} + \mathrm{precision}}
\end{equation}

Similarly, for the negative class. How many of the negative class did we get:

\begin{equation}
\text{specificity}
 = \frac{T\kern-.5exN}{N}
\end{equation}


Recall and specificity can be combined into:


\begin{equation}
\text{Balanced Accuracy }
 = \frac{1}{2} (\mathrm{recall} + \mathrm{specificity})
\end{equation}

``F1 score'' and ``Balanced Accuracy'' are common choices for assessing performance of a classifier evaluated on imbalanced data with different implications with regards to how ``consistent'' this imbalance is. 

}

\end{frame}

\subsection{Further considerations}

\begin{frame}\frametitle{\subsecname}

\begin{enumerate}
\item
Is there only one recall value for my classifier that predicts with\\
 $y(\vec x) \in \lbrack0,1\rbrack$ (or $\lbrack-1,1\rbrack$, $(0,1)$)?
\item
What if the proportions in my training set is different from that of the validation/test set?
\end{enumerate}


\end{frame}

\subsubsection{Calibration}

\begin{frame}\frametitle{\subsubsecname}

\mode<article>{Calibrating the classifier: }
Finding an operating point by adjusting the threshold that converts the prediction of classifier into a hard decision.

General procedure:
\begin{enumerate}
\item train binary classifier (assuming $y(\vec x) \in \lbrack0,1\rbrack$)
\pause
\item make predictions on a \emph{hold-out} set (\notesonly{selecting an operating point}\slidesonly{this} counts as a hyper-parameter selection)
\pause
\item Save ``probabilistic'' output, no thresholding.
\pause
\item for threshold $\mathrm{thr} \in \lbrack0,1\rbrack$ do:
\begin{enumerate}
	\item assign predictions to classes using current threshold $\mathrm{thr}$
\pause
	\item compute confusion matrix (e.g. $T\kern-.4exP$ and $F\kern-.5exP$)
	\item[]repeat
\end{enumerate}
\item Compute metrics as a function of $\mathrm{thr}_i$
\end{enumerate}

\question{What does $thr$ effectively represent?}

\notesonly{
- Effecitvley the bias of the output neuron. Changing $thr$ corresponds to shifting/translating the hyperplane that divides the two classes.
The orientation of the hyperplane remains the same.
}


\end{frame}

\begin{frame}
Example for the calibration procedure: \emph{Receiver operating characteristics (ROC)}. Keep track of $T\kern-.4exP$ and $F\kern-.4exP$ as a function of different threshold values.

\begin{figure}[h]
    \centering
	\includegraphics[width=0.35\textwidth]{img/curves_roc}
	\caption{ROC curve. $T\kern-.4exP/P\,\corresponds $ true positive rate (TPR) and $F\kern-.4exP/P\,\corresponds$ false positive rate (FPR)}
\end{figure}

\only<1>{
The higher the area under curve (ROC AUC) the better the classifier.

\question{What does the dashed diagonal line represent?}

}
\only<2>{

\textbf{But} at the end of the day I need to pick an operating point \notesonly{(i.e. one threshold value)}.

\question{What criterion could I use to favor one threshold over another?}
}

\end{frame}

\mode<article>{
- Assign a ``decision cost $\$_{ij}$'' for different classes 
where $\$_{ij}$ denotes the cost of predicting class $C_{i}$ when the true label is $C_{j}$
(cf. lecture slides 1.4)
}

\begin{frame}

What if the proportions in my training set is different from that of the validation/test set? Which metrics are robust to changes in proportions between different splits of the data.\footnote{cf. jupyter notebook on ISIS for a comparison of different metrics under different conditions.}

\slidesonly{\vspace{-2mm}}

\begin{figure}[h]
	\includegraphics[width=0.8\textwidth]{img/compare_metrics}
	\mode<article>{
	\caption{Comparing metrics under different conditions}
	}
\end{figure}



\end{frame}

\begin{frame}

All we did so far is measure generalization performance and tune our operating point.

\question{Is there anything else we can do about imbalanced data?}

\pause

\begin{enumerate}
\item Train the classifier on a synthetically balanced dataset by
\begin{itemize}
\item sub-sampling from the majority class (no one likes to throw away data)
\item oversampling the minority class
\end{itemize}

\question{Any downsides to oversampling?}

\pause

- the resulting dataset could be too large.
\item class weighted loss function:
e.g. weighted cross entropy:
\begin{equation}
e^{(\alpha)} = -\, \gamma \, y_T^{(\alpha)} \cdot \ln \lbrack y^{(\alpha)} \rbrack - \beta \, (1-y_T^{(\alpha)}) \cdot \ln \lbrack 1-y^{(\alpha)} \rbrack
\end{equation}
where $\beta$ and $\gamma$ represent ``class weights''
\end{enumerate}


\end{frame}




\mode*

\mode<all>
\section{Non-parametric classification}

\subsection{The setting}

\begin{frame}\frametitle{\subsecname for M-way classification}

\mode<article>{
Specifying the data and model for a multi-class classification with $M$ classes (i.e. $M$-way classification)
}

\begin{itemize}
	\item \underline{Data}:\\

	\begin{equation*}
	\Big\{ \left(\vec x^{(\alpha)}, \vec y^{(\alpha)}_{T} \right) \Big\}\,
	\end{equation*}

	where 
	\begin{itemize}
	\item[] $\alpha = 1,\ldots,p$ and
	\item[]$\vec y_T^{(\alpha)} \in \{0, 1\}^M$ with $\sum_{c=1}^{M} (y_{T})_c = 1$ (one-hot encoding of class labels),
	\end{itemize}

	\pause

	\item \underline{Model}:\\

	\begin{equation*}
	\vec y(\vec x) \in \R^M 
	\end{equation*}
	with 
	\begin{itemize}
	\item[] $\sum_{c=1}^{M} y_c(\vec x) = 1$ and
	\item[] $y_c(\vec x)\,\ge\,0\; \forall c$
	\end{itemize}
	
	Use $\vec y(\vec x)$ to give
	\begin{itemize}
	\item probabilities of the predicted class (e.g. $y_5(\vec x) = 0.75\; \leadsto$ class ``5'' with 75\% probability.
	\item hard predictions/decisions: $\argmax_{c=1,\ldots,M} y_c(\vec x)$
	\end{itemize}

\end{itemize}

\end{frame}

\subsection{k nearest neighbor}


\begin{frame}\frametitle{\subsecname}

Prediction follows the majority vote of the $k$ nearest neighbors around the query point.

\begin{figure}[ht]
     \centering
	\includegraphics[width=0.2\textwidth]{img/section4_fig11_K2}
     \mode<article>{
	\caption{Basic RNN architecture}
	}
	\label{fig:rnn} 
\end{figure}

\end{frame}

\begin{frame}\frametitle{\subsecname}

Let $k$NN$(\vec x)$ be the indices $\{\beta_1, \beta_2,\ldots,\beta_k\}$ of the $k$ data points closest to $\vec x$ w.r.t. Euclidean norm:

\begin{equation}
\beta_j = 
\argmin_{\alpha \in \{1,\ldots,p\} \textbackslash \{\beta_{1},\ldots,\beta_{j-1}\}}
\lVert \vec x^{(\alpha)} - \vec x\rVert_{2}
\end{equation}

The $k$NN classifier (for fixed $k~\corresponds$ hyperparameter) is defined by

\begin{equation}
\vec y(\vec x) \frac{1}{k} \sum_{\beta \in k\mathrm{NN}(\vec x)} \vec y_{T}^{(\beta)}
\label{eq:knn_classifier}
\end{equation}

$\vec y(\vec )$ in \eqref{eq:knn_classifier} is effectively an arithmetic average of the labels in the neighborhood.

\end{frame}

\begin{frame}
Example k=3, M=4

\begin{equation}
\vec y(\vec x) = \frac{1}{3}
\left\lbrack
\right\rbrack
\end{equation}
    
\end{frame}

\mode*

\clearpage

\mode<all>


\subsection{Parzen window classification}


\begin{frame}\frametitle{\subsecname}

Prediction is based on 
\begin{itemize}
\item \underline{all} data points \notesonly{(no longer restricted to the $k$ nearest neihbors)}
\item \notesonly{and the contribution of each point is} weighted according to some window/``kernel'' function $\kappa(\vec x, \vec x')$ \notesonly{evaluated for a pair of points $\vec x$ and $\vec x'$}.
\end{itemize}

\begin{equation}
\vec y(\vec x) := \frac{1}{Z} \sum_{\alpha=1}^{p} \vec y_T^{(\alpha)} \cdot \kappa(\vec x, \vec x^{(\alpha)})
\end{equation}

with

\begin{equation}
Z := \sum_{\alpha=1}^{p} \kappa(\vec x, \vec x^{(\alpha)})
\end{equation}

\end{frame}

\begin{frame}

An example for the window/``kernel'' function $\kappa(\vec x, \vec x')$ would be the gaussian function:

\begin{equation}
\kappa(\vec x, \vec x') = \exp\left( -\,\frac{\lVert \vec x - \vec x'\rVert^2_2}{2\,\sigma_{\kappa}^2} \right)
\label{eq:gauss_kernel}
\end{equation}

where $\sigma_{\kappa}$ is referred to as the \emph{width} of the kernel.

\begin{figure}[ht]
     \centering
     \savebox{\imagebox}{
	 \includegraphics[width=0.37\textwidth]{img/guassian_function_1d}}%
     \begin{subfigure}[t]{0.37\textwidth}
         \centering
         \usebox{\imagebox}% Place largest image
         \caption{For data in 1D}
         \label{fig:quadratic}
     \end{subfigure}
     \hspace{2mm}
     \begin{subfigure}[t]{0.37\textwidth}
         \centering
         \raisebox{\dimexpr.5\ht\imagebox-.5\height}{% Raise smaller image into place
         \includegraphics[width=0.99\textwidth]{img/guassian_function_2d}
         }
         \caption{For data in 2D}
         \label{fig:linear}
     \end{subfigure}
     \mode<article>{
     \caption{The gaussian kernel function}
     }
	 \label{fig:quadratic_density_gaussian}
\end{figure}

\end{frame}

\begin{frame}

Binary classification example with 2D data:

\begin{figure}[ht]
     \centering
     \savebox{\imagebox}{
	 \includegraphics[width=0.37\textwidth]{img/parzen_data}}%
     \begin{subfigure}[t]{0.37\textwidth}
         \centering
         \usebox{\imagebox}% Place largest image
         \caption{For data in 1D}
         \label{fig:quadratic}
     \end{subfigure}
     \hspace{2mm}
     \visible<2>{
     \begin{subfigure}[t]{0.37\textwidth}
         \centering
         \raisebox{\dimexpr.5\ht\imagebox-.5\height}{% Raise smaller image into place
         \includegraphics[width=0.99\textwidth]{img/parzen_circles}
         }
         \caption{For data in 2D}
         \label{fig:linear}
     \end{subfigure}
     }
\end{figure}


\mode<presentation>{

\begin{equation}
\vec y(\vec x) := \frac{1}{Z} \sum_{\alpha=1}^{p} \vec y_T^{(\alpha)} \cdot \kappa(\vec x, \vec x^{(\alpha)})
\end{equation}

\begin{equation}
\kappa(\vec x, \vec x') = \exp\left( -\,\frac{\lVert \vec x - \vec x'\rVert^2_2}{2\,\sigma_{\kappa}^2} \right)
\label{eq:gauss_kernel}
\end{equation}

}

\end{frame}

\begin{frame}

\begin{figure}[ht]
     \centering
     \savebox{\imagebox}{
	 \includegraphics[width=0.37\textwidth]{img/parzen_circles}}%
     \begin{subfigure}[t]{0.37\textwidth}
         \centering
         \usebox{\imagebox}% Place largest image
         \caption{wide kernel width}
         \label{fig:quadratic}
     \end{subfigure}
     \hspace{2mm}
     \begin{subfigure}[t]{0.37\textwidth}
         \centering
         \raisebox{\dimexpr.5\ht\imagebox-.5\height}{% Raise smaller image into place
         \includegraphics[width=0.99\textwidth]{img/parzen_circles_narrow}
         }
         \caption{narrow kernel width}
         \label{fig:linear}
     \end{subfigure}
\end{figure}

\question{What is the role of the kernel width $\sigma_{\kappa}$ in terms of model complexity?}

\pause

Hint: Jitter the query points and see which choice of $\sigma_{\kappa}$ will lead to different predictions.

\end{frame}

\begin{frame}

Remarks:
\begin{enumerate}
\item The width of the kernel $\kappa$ controls the model complexity (narrow width $\leadsto$ overfitting (high variance), wide $\leadsto$ underfitting)
 
One way to go about this is to apply a small noise to the query points (i.e. jitter) and see how the predictions will vary with different $\sigma_{\kappa}$. If this small noise in the position of the points leads to different predictions we are bascially looking at high variance.

\item Using all data points. Potentiially costly with large $p$
\end{enumerate}
\end{frame}

\mode*

\clearpage

\mode<all>


\section{RBF-Networks}


\begin{frame}\frametitle{\secname}

A tradeoff between $k$NN classification and Parzen window classification.

Instead of taking all $p$ points to predict the class probabilities (Parzen), limit the prediction to $k \ll p$ \cancel{points} ``representative'' functions.

Represent all points $\vec x$ in terms of basis functions $\phi_i(\vec x), i=1\,\ldots,k$

Let $k$ be the number of basis functions.

\end{frame}

\begin{frame}

\begin{figure}[ht]
     \centering
     \savebox{\imagebox}{
	 \includegraphics[width=0.37\textwidth]{img/parzen_data}}%
     \begin{subfigure}[t]{0.37\textwidth}
         \centering
         \usebox{\imagebox}% Place largest image
         \caption{Two-class data in 2D}
         \label{fig:quadratic}
     \end{subfigure}
     \hspace{2mm}
     \begin{subfigure}[t]{0.37\textwidth}
         \centering
         \raisebox{\dimexpr.5\ht\imagebox-.5\height}{% Raise smaller image into place
         \includegraphics[width=0.99\textwidth]{img/rbf-network}
         }
         \caption{$k=3$ ``representatives''}
         \label{fig:rbf-network}
     \end{subfigure}
\end{figure}

\end{frame}

\subsection{Recap: Regression on transformed data}

\begin{frame}\frametitle{\subsecname}



\end{frame}

\mode*

\clearpage


\end{rightcolumn}
\end{paracol}

\end{document}
