\section{Results statistical learning theory}

\begin{frame}\frametitle{Results of Statistical Learning Theory}
		\begin{itemize}
			\item formulation of conditions under which ERM works and its convergence \notesonly{(cf. \sectionref{sec:convergence_erm})}
			\item bounds describing \textbf{generalization ability} of ERM \\
            
            For a finite training set, the generalization error is bound with probability $1-\epsilon$:
            \begin{equation}
				E^{G}_{[\vec w]} \;\le\; E^{T}_{[\vec w]} + \sqrt{\frac{\dvc \left(\ln \frac{2\,p}{\dvc} + 1\right)-\ln\frac{\epsilon}{4}}{p}} +\frac{1}{p} \quad \text{for } p > \dvc
				\label{eq:generalizationresults}
            \end{equation}
            
            \mode<article>{
            where $\dvc \left(\ln \frac{2\,p}{\dvc} + 1\right) =: G^{\Lambda}_{(2p)}$ from \eqref{eq:growthpiecewise} for $p > \dvc$.
            }
            
            \mode<article>{
            \eqref{eq:generalizationresults} is obtained by solving \eqref{eq:boundgenerlization} for $\epsilon$ \footnote{If interested, cf. supplementary material ``Deviations from the optimal model''}.
            }
            
            100\% overfitting for $p\,<\,\dvc$.
            
            Bound allows calculating $p$ to target a given generalization error.
            
			\item inductive inference for \textbf{small sample size}s 
				based on these bounds
			\item methods for implementing this new type 
				of inference ($\rightarrow$ maximum margin classifiers e.g. {SVMs})
		\end{itemize}
\end{frame}
