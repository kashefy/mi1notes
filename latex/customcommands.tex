% ---------------------------- README -----------------------------------------
% This file holds additional command definitions as used in the Machine 
% Intelligence material.
% -----------------------------------------------------------------------------

% ------------------------- New Commands --------------------------------------
% strikout / strikethrough
\newcommand\hcancel[2][black]{\setbox0=\hbox{$#2$}%
\rlap{\raisebox{.2\ht0}{\textcolor{#1}{\rule{\wd0}{1pt}}}}#2} 

% horizontal bar to put in matrices
\newcommand*{\horzbar}{\rule[.5ex]{2.5ex}{0.5pt}}

% wider strikethrough
\newcommand{\wcancel}[1]{\cancel{\;#1\;}}

\newcommand{\bs}[1]{\boldsymbol{#1}}

\newcommand{\gap}[0]{\vspace{.6cm}}
\newcommand{\ve}[1]{\bm{#1}}

% Expectation/mean over variable
\newcommand{\mean}[1]{\langle #1 \rangle}

% Equal sign with exclamation mark on top:
\newcommand{\eqexcl}{\ensuremath{\stackrel{\mathrm{!}}{=}}}
%			--------------------
% Equal sign with hat on top:			
\newcommand{\corresponds}{\ensuremath{\widehat{=}}}
%			--------------------
% Colon before an equal sign - defined to avoid usage of txfonts: 
\newcommand{\coloneqq}{\ensuremath{:=}}
\newcommand{\eqqcolon}{\ensuremath{=:}}

%			--------------------
% \argmin like \min (operator with limits):
\newcommand{\argmin}{\operatornamewithlimits{argmin}}
%			--------------------
% \argmax as an operator:
\newcommand{\argmax}{\operatornamewithlimits{argmax}}
%			--------------------
% \sign as an operator:
\newcommand{\sign}{\operatorname{sign}}
%			--------------------
% \emp as an operator:
\newcommand{\emp}{\operatorname{emp}}
%			--------------------
% \dvc as a shorthand for the VC-dimension:
\newcommand{\dvc}{\mathrm{d}_{\mathrm{VC}}}
% 			--------------------
% \smallfrac is a \frac in \textstyle
\newcommand{\smallfrac}[2]{ {\textstyle \frac{#1}{#2}} }
% 			--------------------
% \smallsum creates a sum in textstyle with small indices
\newcommand{\smallsum}[2]{ 
	{\textstyle \sum\limits_{\scriptscriptstyle #1}^{\scriptscriptstyle #2}} }
%			--------------------
% Font to denote set names
\newcommand{\Set}[1]{\mathcal{#1}}
%			--------------------
\newcommand{\LLR}{\mathcal{L}}
% KL divergence
\newcommand{\dkl}{\mathrm{D}_{\mathrm{KL}}}
\newcommand{\var}{\operatorname{var}}
% Kurtosis
\newcommand{\kurt}{\operatorname{kurt}}
%			--------------------
\newcommand{\oident}{\rule{5mm}{0pt}}
\newcommand{\normal}{\mathcal{N}}
%			--------------------
% Mathematical standard sets
\newcommand{\R}{\mathbb{R}}
\newcommand{\N}{\mathbb{N}}
\newcommand{\E}{\mathbb{E}}
%			--------------------
% The following three new commands are here to simplify the extensive use
% of specially itemized lists. They are solely intended to be used in the
% itemize environment. 
\newcommand{\itr}{\item[$\rightarrow$]}
\newcommand{\itR}{\item[$\Rightarrow$]}
\newcommand{\itl}{\item[$\leadsto$]}
\newcommand{\iitem}[1]{\begin{itemize} \item {#1} \end{itemize}}
%			--------------------
% To comply with the standard way in neural network textbooks we render all
% vectors with \underline{\mathbf{NAME}}. As this is quite long and can lead
% to unnecessary clutter the \vec{NAME} command was redefined.
\renewcommand{\vec}[1]{\ensuremath{\underline{\mathbf{#1}}}}
% Note: As a standard the \vec{NAME} command normally renders NAME with an 
%	arrow above it (as it is written in more math oriented books). 
%	The \underline{\mathbf{NAME}} method is more common in the field
%	of neural networks and it has the benefit of not interfering and
%	being visually more pleasant when dealing with constructs like
%	\hat{\underline{\mathbf{w}}}.
%	If the 'standard' math mode is again in fashion, just comment.

% Matrices

% matrix [ ]
\newcommand{\mat}[2][rrrrrrrrrrrrrrrrrrrrrrrrrr]{\left[
	\begin{array}{#1}
	#2\\
	\end{array}
	\right]}

% ( )
\newcommand{\rmat}[2][rrrrrrrrrrrrrrrrrrrrrrrrrr]{\left(
	\begin{array}{#1}
	#2\\
	\end{array}
	\right)}

% |[ ]|
\newcommand{\dmat}[2][rrrrrrrrrrrrrrrrrrrrrrrrrr]{\left|\left[
	\begin{array}{#1}
	#2\\
	\end{array}
	\right]\right|}

% |( )|
\newcommand{\drmat}[2][rrrrrrrrrrrrrrrrrrrrrrrrrr]{\left|\left(
	\begin{array}{#1}
	#2\\
	\end{array}
	\right)\right|}

% -----------------------------------------------------------------------------
\newcommand{\placeimage}[4]{
	\begin{textblock}{}(#1,#2)
		\includegraphics[#4]{#3}
	\end{textblock}
}
% -----------------------------------------------------------------------------
\newcommand\blfootnote[1]{%
  \begingroup
  \renewcommand\thefootnote{}\footnote{#1}%
  \addtocounter{footnote}{-1}%
  \endgroup
}

%\newcommand{\url}[1]{{\sc\footnotesize #1}}
\newcommand{\ttopic}[1]{\textbf{\LARGE #1}}
\newcommand{\filename}[1]{\texttt{#1}}

\newcommand{\farrow}{$\Rightarrow\;$}
\newcommand{\myhref}[2]{{\tt\href{#1}{#2}}}

% --------------------------- Aliases -----------------------------------------

% independent and identically distributed
\newcommand{\iid}{i.i.d.}

% parameterized cost function
\newcommand{\tyxw}{\left(y_T, y\left(\vec x; \vec w\right)\right)}
\newcommand{\tyxwalpha}{\left(y^{(\alpha)}_T, y\left(\vec x^{(\alpha)}; \vec w\right)\right)}
\newcommand{\txw}{\left(y_T, \vec x; \vec w\right)}
\newcommand{\txwalpha}{\left(y^{(\alpha)}_T, \vec x^{(\alpha)}; \vec w\right)}

% training error error
\newcommand{\ET}{{E}^T}
\newcommand{\ETw}{\ET_{[\vec w]}}

% generalization error
\newcommand{\EG}{{E}^G}
\newcommand{\EGw}{{E}^G_{[\vec w]}}
\newcommand{\estEG}{\widehat{E}^G}

% message
\newcommand{\msg}[2]{\mu_{#1\rightarrow#2}}
